\chapter*{Введение}
\addcontentsline{toc}{chapter}{Введение}
\label{ch:intro}

В последнее время роботы стали важной частью нашей жизни. Они используются во многих областях и применяются для различных целей, в которых позволяют повысить эффективность, быстродействие и безопасность. 

Самым простым способом внедрения роботов для выполнения задач является использование стационарных роботов. Основываясь на требованиях высокой надежности, они статичны, неинтерактивны и работают в неизменяющихся условиях вдали от людей. Работа данных роботов довольна предсказуема, так как они управляются заранее синтезированными программами для выполнения спланированных задач.

Очевидно, что область применения стационарных роботов ограничена и для многих задач они не смогут быть применены. Соответственно появляется необходимость в мобильных роботах, которые могут осуществлять движение в среде и функционировать с объектами. В связи с этим при работе мобильных роботов задача навигации имеет важное значение. Роботу необходимо обрабатывать информацию об окружении и стоить верные безопасные маршруты в реальном времени для достижения заданных целей. 

Так как мобильный робот функционирует в динамической среде в непосредственной близости от людей, оборудования и других объектов, робот должен достаточно быстро реагировать на изменения и не допускать возникновение критических ситуаций. Для реализации такой высокоуровневой стратегии управления используются алгоритмы принятия решений.

Основная задача данной работы - разработка алгоритма принятия решения для задачи навигации мобильным роботом при функционировании в динамическом окружении. Разработанный алгоритм должен обеспечивать безопасное и эффективное поведение робота.

\endinput