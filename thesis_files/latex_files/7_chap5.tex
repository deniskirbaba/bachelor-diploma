\chapter{Заключение}
\label{ch:chap5}

В данной работе продемонстрирована реализация системы принятия решений для задачи автономной навигации в условиях динамического окружения. Целью разработки системы являлась увеличение автономности, надежности и безопасности. Этого удалось достичь за счет внедрения системы контроля и системы учета динамических объектов. 

Реализация системы контроля существенно снизила количество сценариев, в которых требуется участие человека-оператора, что говорит о более устойчивой и автономной работе. Реализованное поведение, реагирующее на сбои различных компонент системы навигации, ситуации столкновения а также контролирующее заряд батареи,  улучшило качество навигации. В основе данной системы лежит поведенческое дерево, которое позволяет безопасно выполнять более сложные модели поведения, поскольку возможные сценарии, мешающие правильному представлению среды, проверялись и обрабатывались до того, как сложные модели поведения выполнялись на основе этого представления. 

Система учета динамических объектов обеспечивает более безопасную навигацию в условиях динамических препятствий. Она позволяет планировщику траектории учитывать информацию о скорости динамических объектов, таким образом система навигации успешно корректирует траекторию, позволяя роботу избежать столкновения.

Кроме того, архитектура данной системы построена по принципам модульности и может быть усовершенствована, не сталкиваясь с проблемами или ограничениями на более поздних этапах разработки.
